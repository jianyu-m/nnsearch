\documentclass{article}
\usepackage{minted}

\makeatletter

\def\v#1{{\fontfamily{cmtt}\selectfont #1}}

\begin{document}
\section{Problem Definition}
Implement a nearest neighbor search technique on grid-indexed data.
The algorithm should take as input: a query location q and an integer k and
finds the k-nearest neighbors of q using the grid index.

\section{Design and Implementation}
\subsection{Grid construction}
Each data record is stored in a data structure named \v{Record}. The grids
are stored in a multi-array for fast accesses. The grid construction algorithm
is showed in Figure \ref{grid}. The construction needs two scan of the dataset.
In the first scan, it finds the maximum latitude and longitude of all location.
Then in the second scan, put the record to its corresponding grid.
\begin{listing}[ht]
\begin{minted}
[
  frame=lines,
  framesep=2mm,
  fontsize=\footnotesize,
  linenos
]{bash}
# visited
# min_lat, min_long, max_lat, max_long are got in the first scan
foreach (r <- records)
  if not r in visited
    x = (r.lat - min_lat) / grid_size
    y = (r.long - min_long) / grid_size
    grids(x)(y) += r
\end{minted}
\caption{Grid construction algorithm.}
\label{grid}
\end{listing}

\subsection{k-nearest neighbour search on grids}
A k-nearest neighbour search query accepts a parameter of
location (latitude and longitude). It first finds the corresponding
grid of the location, and then find a k-nearest locations in the grid.
After that, it searches the layered cells. A function called
\v{grid\_d} (see Figure \ref{gridd}) is to search inside a grid. \v{k\_set} is list with k
records ordered by its distance with the location. The full algorithm is showed in
Figure \ref{knn}.

\begin{listing}[ht]
\begin{minted}
[
  frame=lines,
  framesep=2mm,
  fontsize=\footnotesize,
  linenos
]{python}
def grid_d(grid_records, lat, long, k_set):
  for (r <- grid_record):
    if r < k_set.max or len(k_set) < k
      k_set.insert(r)
  rerurn k_set
\end{minted}
\caption{Function grid\_d.}
\label{gridd}
\end{listing}

\begin{listing}[ht]
\begin{minted}
[
  frame=lines,
  framesep=2mm,
  fontsize=\footnotesize,
  linenos
]{python}
def (lat, long, k):
  x = (lat - min_lat) / grid_size
  y = (lat - min_long) / grid_size
  k_set = grid_search(grids(x)(y), k_set)
  while (true):
    min_d
    for (cell <- cells layered around x, y):
      d = min(cell, lat, long)
      min_d = min(d, min_d)
      if d < k_set.min or len(k_set) < k:
        k_set = grid_search(cell, k_set)
    if (len(k_set) >= k and k_set.max < d_min):
      break
  return k_set
\end{minted}
\caption{K-Nearest Neighbours algorithm.}
\label{knn}
\end{listing}
In this algorithm, if the minimum distance between the cell and
the location is larger than the maximum value in \v{k\_set},
then the cell does not need to be access. This algorithm stops
when the minimum distance between the location and all cells in the layer
(none of the cells in the layer has been accessed).

\subsection{benchmark}
Evaluation of the algorithm is focused of the time of constructing the grids
and the average query time. Locations were generated randomly and
these locations are served as queries to the knn search system. The average
query time for 10k queries is recorded. A simple linear scan algorithm is
used for comparison. The result is showed in Table \ref{result}.

\begin{table}[tbh]
  \center
  \footnotesize
  \begin{tabular}{c|c|c|c}
    \textbf{name} & \textbf{construction time} & \textbf{query time} & \textbf{cell accessed}\\
    \hline
    grid search(100 x 100) & 22s & 3ms & 1477 \\
    \hline
    grid search(100 x 50) & 22s & 6ms & 799 \\
    \hline
    grid search(50 x 100) & 22s & 1ms & 380 \\
    \hline
    linear scan & N/A & 32ms & N/A
  \end{tabular}
  \caption{Evaluation result.}
  \label{result}
\end{table}

Table \ref{result} shows that linear scan has a much larger access time compared with
grid search. This is because grid search can avoid unnecessary search. 50 x 100 grid
has a better performance as it better suits the range of the train data.

\end{document}
